\begin{frame}{The problem of reproducibility for Imaging genetics}

\begin{block}{JB Poline}

Helen Wills Neuroscience Institute, UC Berkeley

\end{block}

\end{frame}

\begin{frame}{Reproducibility - preliminary remarks}

\begin{itemize}
\itemsep1pt\parskip0pt\parsep0pt
\item
  Reminding ourselves : Reproducibility is the backbone of scientific
  activity
\item
  Reproducibility versus replicability
\item
  Is there a problem ?
\item
  Not everybody is convinced that there is a problem
\item
  do we have hard evidence ?
\end{itemize}

\end{frame}

\begin{frame}{Reproducibility - evidence of the problem}

\begin{itemize}
\itemsep1pt\parskip0pt\parsep0pt
\item
  In epidemiology
\item
  In social sciences and in psychology
\item
  In cognitive neuroscience
\item
  In brain imaging
\item
  In genetics
\item
  And not in imaging genetics ?
\end{itemize}

.. Ioannidis 2005 .. Reproducibility Project: Psychology
(https://osf.io/ezcuj/wiki/home/) .. Barch, Deanna M., and Tal Yarkoni.
``Introduction to the Special Issue on Reliability and Replication in
Cognitive and Affective Neuroscience Research.'' Cognitive, Affective,
\& Behavioral Neuroscience 13, no. 4 (December 2013): 687--89.
doi:10.3758/s13415-013-0201-7. .. Functional and Structural
Neuroimaging: Reproducibility Issues in Multicentre MRI Studies, Jorge
Jovicich, Univ. of Trento .. Hunter, David J., and Peter Kraft.
``Drinking from the Fire Hose---statistical Issues in Genomewide
Association Studies.'' N Engl J Med 357, no. 5 (2007): 436--39. ..

\end{frame}

\begin{frame}{Why do we have a problem?}

\begin{itemize}
\itemsep1pt\parskip0pt\parsep0pt
\item
  Things are getting complex
\item
  Publication pressure is high
\item
  Mistakes are done
\item
  Power issues
\end{itemize}

\end{frame}

\begin{frame}{Things are getting complex}

\begin{itemize}
\itemsep1pt\parskip0pt\parsep0pt
\item
  data complexity (eg: chip idiosynchrasis, format, preprocessings, etc)
\item
  data need to be linked appropriately
\item
  data size: number of variables - files you cannot check visually
\item
  methods: we have to trust external software
\item
  methods: complexity higher
\end{itemize}

\end{frame}

\begin{frame}{Publication pressure is high}

\begin{itemize}
\itemsep1pt\parskip0pt\parsep0pt
\item
  There's no way there isnt a paper out of this data set.
\item
  You will not get your Phd if you don't publish this study
\item
  You won't get tenure
\item
  You won't get funding and your peers admiration and consideration
\item
  Conclusion: the pressure is \emph{very} high
\end{itemize}

\end{frame}

\begin{frame}{Mistakes are done}

The ``Mistakes'' argument : an unpopular topic.

\begin{itemize}
\itemsep1pt\parskip0pt\parsep0pt
\item
  Ioannidis 2005
\item
  Anatomy of an Error
\item
  The Left/Right issue
\item
  The ADHD 1000 connectome
\end{itemize}

\end{frame}

\begin{frame}{The power issue}

\begin{itemize}
\itemsep1pt\parskip0pt\parsep0pt
\item
  show Ioannidis figure
\item
  show Button figure
\end{itemize}

\begin{figure}[htbp]
\centering
\includegraphics{./images/dono.png}
\caption{donoho}
\end{figure}

\end{frame}

\begin{frame}{What is specific to Imaging Genetics}

\begin{itemize}
\itemsep1pt\parskip0pt\parsep0pt
\item
  Combinaison of imaging and of genetics issues: ``AND'' (if
  independant: would probability of getting it rigt would multiply: .7 *
  .7 = .5)
\item
  The combinaison of having to get very large number of subjects for
  GWAS and not being able to get them in imaging
\item
  The ``trendiness'' of the field
\item
  The rapidity of the changes
\item
  The capacity to ``rationalize findings'' (eg: noise in brain images is
  always interpretable)
\end{itemize}

\end{frame}

\begin{frame}{What are the solutions: technical}

\begin{itemize}
\itemsep1pt\parskip0pt\parsep0pt
\item
  Pre-register hypotheses
\item
  Statistics:

  \begin{itemize}
  \itemsep1pt\parskip0pt\parsep0pt
  \item
    Always try to get a sense of the power
  \item
    Take the right statistical tool
  \item
    meta analysis if you can
  \item
    replication always
  \item
    Effect size variation estimation (bootstrapping)
  \end{itemize}
\end{itemize}

\end{frame}

\begin{frame}{What are the solutions: learning}

\begin{itemize}
\itemsep1pt\parskip0pt\parsep0pt
\item
  Learn the right tools:

  \begin{itemize}
  \itemsep1pt\parskip0pt\parsep0pt
  \item
    how can I check my code ? How can I go back to a certain state ?
    (learn git/mercurial, learn git Annex or others)
  \item
    How can others check my analyses? Learn the emerging social open
    science frameworks
  \end{itemize}
\item
  Learn ``one layer below'' (A. Martelli)
\end{itemize}

\end{frame}

\begin{frame}{Train the new generation}

\begin{itemize}
\itemsep1pt\parskip0pt\parsep0pt
\item
  statistics: in depth
\item
  computing: in depth
\item
  a more collaborative and a more open science model
\item
  Work such that the next post-doc will need weeks to start progress -
  not months
\item
  Work such that others in the community can reproduce \textbf{and}
  build upon
\end{itemize}

\end{frame}

\begin{frame}{What are the solutions: social}

\begin{itemize}
\itemsep1pt\parskip0pt\parsep0pt
\item
  Put some pressure on editors to

  \begin{itemize}
  \itemsep1pt\parskip0pt\parsep0pt
  \item
    accept replication studies
  \item
    accept preregistration
  \item
    increase the verifiability of analyses (code and data available)
  \end{itemize}
\item
  Share data / share intermediate results
\item
  increase the capacity of the community to verify
\item
  increase capacity to do meta/mega analyses
\item
  because we should be more interested in replication and new findings
  than our own publication record
\item
  Put pressure to change their evaluation criteria - Decrease
  publication pressure
\end{itemize}

\end{frame}
